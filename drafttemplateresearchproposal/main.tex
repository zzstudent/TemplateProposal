%%%%%%%%%%%%%%%%%%%%%%%%%%%%%%%%%%%%%%%%%
% Research project proposal template
% Based on:
%
% LaTeX Template
% Version 2.5 (27/8/17)
%
% This template was downloaded from:
% http://www.LaTeXTemplates.com
%
% Version 2.x major modifications by: 
% Helen Robertson
%
% With thanks to:
% Matthew Woolway and Terence Van Zyl for help with coding and content.
%
% This template is based on a template by:
% Steve Gunn (http://users.ecs.soton.ac.uk/srg/softwaretools/document/templates/)
% Sunil Patel (http://www.sunilpatel.co.uk/thesis-template/)
%
% Template license:
% CC BY-NC-SA 3.0 (http://creativecommons.org/licenses/by-nc-sa/3.0/)
% 
% This template has been constructed in accordance with the requirements and conventions of the School of Computer Science and Applied Mathematics and of the Faculty of Science at the University of the Witwatersrand.
%
%%%%%%%%%%%%%%%%%%%%%%%%%%%%%%%%%%%%%%%%%

%----------------------------------------------------------------------------------------
%	PACKAGES AND OTHER DOCUMENT CONFIGURATIONS
%----------------------------------------------------------------------------------------

\documentclass[
12pt, % The default document font size, options: 10pt, 11pt, 12pt
oneside, % Two side (alternating margins) for binding by default, uncomment to switch to one side
english, % ngerman for German
onehalfspacing, % One-and-a-half line spacing, alternatives: singlespacing or doublespacing
%draft, % Uncomment to enable draft mode (no pictures, no links, overfull hboxes indicated)
nolistspacing, % If the document is onehalfspacing or doublespacing, uncomment this to set spacing in lists to single
liststotoc, % Uncomment to add the list of figures/tables/etc to the table of contents
%toctotoc, % Uncomment to add the main table of contents to the table of contents
%parskip, % Uncomment to add space between paragraphs
%nohyperref, % Uncomment to not load the hyperref package
headsepline, % Uncomment to get a line under the header
%chapterinoneline, % Uncomment to place the chapter title next to the number on one line
%consistentlayout, % Uncomment to change the layout of the declaration, abstract and acknowledgements pages to match the default layout
]{ProposalAndThesis} % The class file specifying the document structure

\usepackage[utf8]{inputenc} % Required for inputting international characters
\usepackage[T1]{fontenc} % Output font encoding for international characters

\usepackage{mathpazo} % Use the Palatino font by default

\usepackage[backend=bibtex,style=numeric,natbib=true]{biblatex} % Use the bibtex backend with the numeric citation style

\addbibresource{example.bib} % The filename of the bibliography

\usepackage[autostyle=true]{csquotes} % Required to generate language-dependent quotes in the bibliography

\usepackage[cleanlook, english]{isodate} % Required for UK date formatting

\usepackage{fancybox} % Required for boxed text sections
\usepackage{xcolor} % Required for coloured text

\usepackage{pgfgantt}

\definecolor{ShadowColor}{RGB}{0,103,165} % Required for coloured shadow in boxed text sections
\makeatletter
\newcommand\Cshadowbox{\VerbBox\@Cshadowbox}
\def\@Cshadowbox#1{%
	\setbox\@fancybox\hbox{\fbox{#1}}%
	\leavevmode\vbox{%
		\offinterlineskip
		\dimen@=\shadowsize
		\advance\dimen@ .5\fboxrule
		\hbox{\copy\@fancybox\kern.5\fboxrule\lower\shadowsize\hbox{%
				\color{ShadowColor}\vrule \@height\ht\@fancybox \@depth\dp\@fancybox \@width\dimen@}}%
		\vskip\dimexpr-\dimen@+0.5\fboxrule\relax
		\moveright\shadowsize\vbox{%
			\color{ShadowColor}\hrule \@width\wd\@fancybox \@height\dimen@}}}
\makeatother

%----------------------------------------------------------------------------------------
%	MARGIN SETTINGS
%----------------------------------------------------------------------------------------

\geometry{
	paper=a4paper, % Change to letterpaper for US letter
	inner=2.8cm, % Inner margin
	outer=2.8cm, % Outer margin
	bindingoffset=0.0cm, % Binding offset
	top=2.8cm, % Top margin
	bottom=2.8cm, % Bottom margin
	%showframe, % Uncomment to show how the type block is set on the page
}

%----------------------------------------------------------------------------------------
%	THESIS INFORMATION
%----------------------------------------------------------------------------------------

\thesistitle{Working Title Here} % Your thesis title, this is used in the title and abstract, print it elsewhere with \ttitle
\supervisor{Name of Supervisor Here} % Your supervisor's name, this is used in the title page, print it elsewhere with \supname
\cosupervisor{Name of Co-supervisor Here (or Delete)} % Your co-supervisor's name, this is used in the title page, print it elsewhere with \cosupname
%\examiner{} % Your examiner's name, this is not currently used anywhere in the template, print it elsewhere with \examname
\degree{Degree Name Here} % Your degree name, this is used in the title page and abstract, print it elsewhere with \degreename
\author{Your Name Here} % Your name, this is used in the title page and abstract, print it elsewhere with \authorname
%\addresses{} % Your address, this is not currently used anywhere in the template, print it elsewhere with \addressname
%\subject{} % Your subject area, this is not currently used anywhere in the template, print it elsewhere with \subjectname
%\keywords{} % Keywords for your thesis, this is not currently used anywhere in the template, print it elsewhere with \keywordnames
\university{University of the Witwatersrand, Johannesburg} % Your university's name, this is used in the title page and abstract, print it elsewhere with \univname
\department{Name of School or Department Here} % Your department's name, this is used in the title page and abstract, print it elsewhere with \deptname
%\group{\href{http://research group.com}{Research Group Name}} % Your research group's name and URL, this is not currently used anywhere in the template, print it elsewhere with \groupname
%\faculty{\href{http://faculty.university.com}{Faculty Name}} % Your faculty's name and URL, this is not currently used anywhere in the template, print it elsewhere with \facname

\def\keywordnames{Appendices; Chapters; Figures; example.bib; main.pdf; main.tex; main.bbl; main.aux; main.blg; main.lof; main.log; main.lot; main.out; MastersDoctoralThesis.cls}

\newcommand{\keyword}[1]{\textbf{#1}}
\newcommand{\tabhead}[1]{\textbf{#1}}
\newcommand{\code}[1]{\texttt{#1}}
\newcommand{\file}[1]{\texttt{\bfseries#1}}
\newcommand{\option}[1]{\texttt{\itshape#1}}



\AtBeginDocument{
\hypersetup{pdftitle=\ttitle} % Set the PDF's title to your title
\hypersetup{pdfauthor=\authorname} % Set the PDF's author to your name
\hypersetup{pdfkeywords=\keywordnames} % Set the PDF's keywords to your keywords
}

\begin{document}

\frontmatter % Use roman page numbering style (i, ii, iii, iv...) for the pre-content pages

\pagestyle{plain} % Default to the plain heading style until the thesis style is called for the body content

%----------------------------------------------------------------------------------------
%	TITLE PAGE
%----------------------------------------------------------------------------------------

\begin{titlepage}
\begin{center}

{\huge \bfseries \ttitle}\par\vspace{0.4cm} % Thesis title
\HRule\par\vspace{1.5cm}
\authorname\par\vspace{1cm}
\emph{Supervisor(s):}\par
{\supname}\par % Name of supervisor
{\cosupname} % Name of co-supervisor
\par\vspace{0.5cm}

\includegraphics[width=80mm]{Figures/logoWitsstackedcolourtransparent.png} % University crest
\vfill

A research proposal submitted in partial fulfillment of the requirements for the degree of \degreename\par\vspace{0.3cm}
in the\par\vspace{0.4cm}
\deptname\par\vspace{0.1cm} % Name of department
\univname\par\vspace{0.4cm} % Name of university
\cleanlookdateon
\today % Date

\end{center}

\end{titlepage}

%---------------------------------------------------------------------------------------
%	USING THIS TEMPLATE
%---------------------------------------------------------------------------------------

\par\vspace{0.5cm}
\noindent \Cshadowbox{
	\begin{minipage}{15cm}
		\medskip
		\color[rgb]{0.0,0.4,0.65} 
		\begin{center} 
			\medskip \textbf{Using this template} \par 
			\end{center}
			\medskip The template below has been constructed in line with the Faculty of Science requirements as well as the conventions of the School. Guidance on each section can be found in the blue boxes at the start of the section. Note that conventions and preferences for structuring a research proposal can vary from discipline to discipline and supervisor to supervisor. Both the template and the guidance on the different sections are suggestions for structuring the proposal. If your supervisor has a different preferred convention or template, it is recommended that you consult with them regarding the differences. \par \medskip
			Further details on using the template can be found in Appendix B. \par \medskip
			You should ensure that you either delete or comment out the blue boxes before submission of the proposal. \par
		\medskip 
\end{minipage}}\\ 

%----------------------------------------------------------------------------------------
%	DECLARATION PAGE
%----------------------------------------------------------------------------------------

\begin{declaration}
\addchaptertocentry{\authorshipname} % Add the declaration to the table of contents
\vspace{0.5cm}
\noindent I, \authorname, declare that this proposal is my own, unaided work. It is being submitted for the degree of {\degreename} at the \univname. It has not been submitted for any degree or examination at any other university.

\par\vspace{2cm}
\begin{flushright}
\includegraphics[width=30mm]{Figures/SignatureSample.png}\par 
\authorname\par\vspace{0.1cm}
\today
\end{flushright}

\par\vspace{0.5cm}
\noindent \Cshadowbox{
	\begin{minipage}{15cm}
		\medskip
		\color[rgb]{0.0,0.4,0.65} 
		\medskip The declaration is an important formal requirement. Ensure that you upload an
		image of your signature and that you change the file name in the main.tex file to include it.
		\medskip 
\end{minipage}}\\ 

\end{declaration}
\vfill
\pagebreak

%----------------------------------------------------------------------------------------
%	QUOTATION PAGE
%----------------------------------------------------------------------------------------

%\vspace*{0.2\textheight}

%\noindent\enquote{\itshape Thanks to my solid academic training, today I can write hundreds of words on virtually any topic without possessing a shred of information, which is how I got a good job in journalism.}\bigbreak

%\hfill Dave Barry

%----------------------------------------------------------------------------------------
%	ABSTRACT PAGE
%----------------------------------------------------------------------------------------

\begin{abstract}
\addchaptertocentry{\abstractname} % Add the abstract to the table of contents
\begin{quote}
Lorem ipsum dolor sit amet, consectetur adipiscing elit. Aliquam ultricies lacinia euismod. Nam tempus risus in dolor rhoncus in interdum enim tincidunt. Donec vel nunc neque. In condimentum ullamcorper quam non consequat. Fusce sagittis tempor feugiat. Fusce magna erat, molestie eu convallis ut, tempus sed arcu. Quisque molestie, ante a tincidunt ullamcorper, sapien enim dignissim lacus, in semper nibh erat lobortis purus. Integer dapibus ligula ac risus convallis pellentesque.
\end{quote}

\par\vspace{0.5cm}
\noindent \Cshadowbox{
    \begin{minipage}{15cm}
    \bigskip
    \color[rgb]{0.0,0.4,0.65}The abstract is a brief informative summary of the proposed research. It can be read independently and should 
    	\begin{itemize}
    		\item locate the proposed research within the background relevant to it,
    		\item state the research question or aim,
    		\item briefly describe the proposed methods for answering the question or achieving the aim, and,
    		\item emphasise the contribution that the proposed research will make to current research in the field.
    	\end{itemize}
    	It is recommended that your abstract is no more than 300 words.
    \medskip
	\end{minipage}}

\end{abstract}

%----------------------------------------------------------------------------------------
%	ACKNOWLEDGEMENTS
%----------------------------------------------------------------------------------------

\begin{acknowledgements}
\addchaptertocentry{\acknowledgementname} % Add the acknowledgements to the table of contents
\vspace{0.5cm}
\noindent{Lorem ipsum dolor sit amet, consectetur adipiscing elit. Aliquam ultricies lacinia euismod. Nam tempus risus in dolor rhoncus in interdum enim tincidunt. Donec vel nunc neque.}

\par\vspace{0.5cm}
\noindent \Cshadowbox{
	\begin{minipage}{15cm}
		\medskip
		\color[rgb]{0.0,0.4,0.65} 
		\medskip The acknowledgements section allows you to thank those who contributed to the
		preparation of the proposal. It is usual to acknowledge supervision, financial
		assistance or funders, and any special facilities provided for the research.		
		\medskip 
\end{minipage}}\\ 



\end{acknowledgements}

%----------------------------------------------------------------------------------------
%	LIST OF CONTENTS/FIGURES/TABLES PAGES
%----------------------------------------------------------------------------------------

\tableofcontents % Prints the main table of contents

\listoffigures % Prints the list of figures

\listoftables % Prints the list of tables

%----------------------------------------------------------------------------------------
%	ABBREVIATIONS
%----------------------------------------------------------------------------------------

%\begin{abbreviations}{ll} % Include a list of abbreviations (a table of two columns)

%\textbf{LAH} & \textbf{L}ist \textbf{A}bbreviations \textbf{H}ere\\
%\textbf{WSF} & \textbf{W}hat (it) \textbf{S}tands \textbf{F}or\\

%\end{abbreviations}

%----------------------------------------------------------------------------------------
%	PHYSICAL CONSTANTS/OTHER DEFINITIONS
%----------------------------------------------------------------------------------------

%\begin{constants}{lr@{${}={}$}l} % The list of physical constants is a three column table

% The \SI{}{} command is provided by the siunitx package, see its documentation for instructions on how to use it

%Speed of Light & $c_{0}$ & \SI{2.99792458e8}{\meter\per\second} (exact)\\
%Constant Name & $Symbol$ & $Constant Value$ with units\\

%\end{constants}

%----------------------------------------------------------------------------------------
%	SYMBOLS
%----------------------------------------------------------------------------------------

%\begin{symbols}{lll} % Include a list of Symbols (a three column table)

%$a$ & distance & \si{\meter} \\
%$P$ & power & \si{\watt} (\si{\joule\per\second}) \\
%Symbol & Name & Unit \\

%\addlinespace % Gap to separate the Roman symbols from the Greek

%$\omega$ & angular frequency & \si{\radian} \\

%\end{symbols}

%----------------------------------------------------------------------------------------
%	DEDICATION
%----------------------------------------------------------------------------------------

%\dedicatory{For/Dedicated to/To my\ldots} 

%----------------------------------------------------------------------------------------
%	THESIS CONTENT - CHAPTERS
%----------------------------------------------------------------------------------------

\mainmatter % Begin numeric (1,2,3...) page numbering

%\pagestyle{thesis} % Return the page headers back to the "thesis" style

% Include the chapters of the thesis as separate files from the Chapters folder
% Uncomment the lines as you write the chapters

\chapter{Introduction} % Main chapter title
\label{Chapter1} % For referencing this chapter elsewhere, use \ref{Chapter1}

\noindent Lorem ipsum dolor sit amet, consectetur adipiscing elit. Aliquam ultricies lacinia euismod. Nam tempus risus in dolor rhoncus in interdum enim tincidunt. Donec vel nunc neque. In condimentum ullamcorper quam non consequat. Fusce sagittis tempor feugiat. Fusce magna erat, molestie eu convallis ut, tempus sed arcu. 

\par\vspace{0.5cm}
\noindent \Cshadowbox{
	\begin{minipage}{15cm}
		\bigskip
		\color[rgb]{0.0,0.4,0.65} The introductory chapter introduces and motivates the proposed research. It should provide the reader with enough information to understand what the proposed research is, how it relates to its broader context, and why it is worthwhile. \par \medskip	
		Throughout the proposal, you should write as if your reader has knowledge of your field broadly, but no more specific or domain knowledge than this. You should try to include as much information as a reader like this would need to understand the research (and no more than this). \par \medskip
		The introduction to the chapter itself should
		\begin{itemize}
			\item introduce the broad area of research/problem area,
			\item introduce the more specific area of research/problem area, 
			\item introduce the research question or problem.
		\end{itemize}
		\medskip
\end{minipage}}

\section{Literature review}

\noindent Sed ullamcorper quam eu nisl interdum at interdum enim egestas. Aliquam placerat justo sed lectus lobortis ut porta nisl porttitor. Vestibulum mi dolor, lacinia molestie gravida at, tempus vitae ligula. Donec eget quam sapien, in viverra eros.

\par\vspace{0.5cm}
\noindent \Cshadowbox{
	\begin{minipage}{15cm}
		\bigskip
		\color[rgb]{0.0,0.4,0.65} The role of a literature review section (or chapter) is, first, to show your familiarity with the research relevant to your proposed research and, secondly, to motivate the proposed research (that is, to provide support for the claim that the research question has not yet been asked or answered or has not yet been adequately asked or answered). \par \medskip		
		The literature review section should 
		\begin{itemize}
			\item begin with a description of the current state of research on the proposed research question,
			\item provide a systematic survey of the literature relevant to the proposed research that motivates it,
			\item conclude by indicating how the proposed research is motivated in light of the survey above.
		\end{itemize}
		 A literature review can be structured in a number of different ways (for example, chronologically, thematically, methodologically). You should choose a structure suitable to motivating your proposed research.
		
		\medskip
\end{minipage}} 

\subsection{Subsection} 

\noindent Lorem ipsum dolor sit amet, consectetur adipiscing elit. Aliquam ultricies lacinia euismod. Nam tempus risus in dolor rhoncus in interdum enim tincidunt. Donec vel nunc neque.

\section{Problem Statement}

\noindent Morbi rutrum odio eget arcu adipiscing sodales. Aenean et purus a est pulvinar pellentesque.Cras in elit neque,  quis varius elit.   Phasellus fringilla,  nibh eu tempus venenatis,  dolor elitposuere quam, quis adipiscing urna leo nec orci.  Sed nec nulla auctor odio aliquet consequat.Ut nec nulla in ante ullamcorper aliquam at sed dolor.  Phasellus fermentum magna in auguegravida cursus. Cras sed pretium lorem. Pellentesque eget ornare odio. Proin accumsan, massaviverra cursus pharetra, ipsum nisi lobortis velit, a malesuada dolor lorem eu neque. 

\par\vspace{0.5cm}
\noindent \Cshadowbox{
	\begin{minipage}{15cm}
		\bigskip
		\color[rgb]{0.0,0.4,0.65} If your research involves identifying a problem, this section will put forward a statement of the problem. A problem statement typically begins with a description of the ideal situation (relevant to the research), describes the current real situation (relevant to the research), and finally states the way in which the proposed
		research will bring the current situation closer to the described ideal situation. If your research does not involve identifying a problem, this section can be omitted.
		 \par	
		\medskip
\end{minipage}} 

\section{Research Question}

\noindent Morbi rutrum odio eget arcu adipiscing sodales. Aenean et purus a est pulvinar pellentesque.Cras in elit neque,  quis varius elit.   Phasellus fringilla,  nibh eu tempus venenatis,  dolor elitposuere quam, quis adipiscing urna leo nec orci.  Sed nec nulla auctor odio aliquet consequat.Ut nec nulla in ante ullamcorper aliquam at sed dolor.

\par\vspace{0.5cm}
\noindent \Cshadowbox{
	\begin{minipage}{15cm}
		\bigskip
		\color[rgb]{0.0,0.4,0.65} This section of the chapter states and discusses in a clear and succinct way the question that the research will seek to answer. The question should be precise, unambiguous, succinct, and sufficiently narrow. If your research involves hypothesis-led methods, this section identifies and discusses the hypothesis of
		the research.
		\par	
		\medskip
\end{minipage}} 


\section{Research Aims and Objectives}

Lorem ipsum dolor sit amet, consectetur adipiscing elit. Aliquam ultricies lacinia euismod. Nam tempus risus in dolor rhoncus in interdum enim tincidunt. Donec vel nunc neque. In condimentum ullamcorper quam non consequat. Fusce sagittis tempor feugiat. Fusce magna erat, molestie eu convallis ut, tempus sed arcu. 

\subsection{Research Aims}

Sed ullamcorper quam eu nisl interdum at interdum enim egestas. Aliquam placerat justo sed lectus lobortis ut porta nisl porttitor. Vestibulum mi dolor, lacinia molestie gravida at, tempus vitae ligula. 

\par\vspace{0.5cm}
\noindent \Cshadowbox{
	\begin{minipage}{15cm}
		\bigskip
		\color[rgb]{0.0,0.4,0.65} This section of the chapter identifies the aim of the research. The ’aim’ of the research is the overarching and broadest goal of the research. Remember that it must be a goal that the research literally aims to achieve. For example, if the problem of the research is a lack of accurate classification models for a certain
		type of image within some domain, the aim of the research cannot be to ’develop a model that will be widely adopted within the domain’. This is something that you as the researcher might hope for, but it is not literally a goal of the research. The (literal) aim of the research would be ’to develop a model for the classification
		of images of type X...’. Remember that the aim is stated using the infinitive. For example, ’the aim of the research is \emph{to develop} a model...’
		
		\par	
		\medskip
\end{minipage}} 


\newpage
\subsection{Objectives}

Sed ullamcorper quam eu nisl interdum at interdum enim egestas. Aliquam placerat justo sed lectus lobortis ut porta nisl porttitor. Vestibulum mi dolor, lacinia molestie gravida at, tempus vitae ligula. Donec eget quam sapien, in viverra eros.

\par\vspace{0.5cm}
\noindent \Cshadowbox{
	\begin{minipage}{15cm}
		\medskip
		\color[rgb]{0.0,0.4,0.65} This section of the chapter identifies the objectives of the research. The 'objectives' of the research are the smaller tasks that are carried out in order to achive your aim. You should identify 4 to 6 objectives. You should ensure that your objectives are precise, measurable, achievable, and directly linked to your research aim. Again, remember that objectives are stated using the infinitive: 'The objectives of the research are \emph{to construct} a data set... etc.’
		\medskip 
\end{minipage}}\\ 
 
\section{Limitations}

Sed ullamcorper quam eu nisl interdum at interdum enim egestas. Aliquam placerat justo sed lectus lobortis ut porta nisl porttitor. Vestibulum mi dolor, lacinia molestie gravida at, tempus vitae ligula. Donec eget quam sapien, in viverra eros.

\par\vspace{0.5cm}
\noindent \Cshadowbox{
	\begin{minipage}{15cm}
		\medskip
		\color[rgb]{0.0,0.4,0.65}This section of the chapter addresses the scope of the research by identifying and discussing its limitations. These limitations might have a number of sources. For example, the size and extent of the data set used for the research could limit its generalisability. In this section, you should identify and discuss any such limitations that are foreseeable at the time of proposal.
		\medskip 
\end{minipage}}\\

%\section{Assumptions and Definitions}
%
%Morbi rutrum odio eget arcu adipiscing sodales. Aenean et purus a est pulvinar pellentesque.Cras in elit neque,  quis varius elit. Phasellus fringilla,  nibh eu tempus venenatis,  dolor elitposuere quam, quis adipiscing urna leo nec orci.  

%\par\vspace{0.5cm}
%\noindent \Cshadowbox{
%	\begin{minipage}{15cm}
%		\bigskip
%		\color[rgb]{0.0,0.4,0.65} Like the section above, this section addresses the scope of the research, along with providing any relevant technical definitions from the literature. It should include the following.
%			\begin{description}
%				\item[Assumptions] Identify and discuss any notable assumptions of the research.
%				\item[Definitions] Define any important technical terms appearing in the research.
%			\end{description}
%		\medskip 
%\end{minipage}}\\

\section{Overview}

Lorem ipsum dolor sit amet, consectetur adipiscing elit. Aliquam ultricies lacinia euismod. Nam tempus risus in dolor rhoncus in interdum enim tincidunt. Donec vel nunc neque. In condimentum ullamcorper quam non consequat. Fusce sagittis tempor feugiat. 

\par\vspace{0.5cm}
\noindent \Cshadowbox{
	\begin{minipage}{15cm}
		\bigskip
		{\color[rgb]{0.0,0.4,0.65}This brief and final section provides an overview of the proposal that follows. Remember to keep this section brief and to-the-point.}
		\medskip 
\end{minipage}}\\

\chapter{Research Methodology}
\label{Chapter2} % For referencing this chapter elsewhere, use \ref{Chapter3}

\noindent Lorem ipsum dolor sit amet, consectetur adipiscing elit. Aliquam ultricies lacinia euismod. Nam tempus risus in dolor rhoncus in interdum enim tincidunt. Donec vel nunc neque. In condimentum ullamcorper quam non consequat. Fusce sagittis tempor feugiat.
\par\vspace{0.5cm}
\noindent \Cshadowbox{
    \begin{minipage}{15cm}
    	\bigskip
    	\color[rgb]{0.0,0.4,0.65}The methodology chapter provides an overview and description of all the important elements of \emph{how} the research will be carried out. It differs from the methodology chapter of the final report insofar as it includes discussion 
    		\begin{itemize}
    			\item both of available and chosen methods, and,
    			\item of any foreseeable limitations of the methods. 
    		\end{itemize}
    	\medskip
    \end{minipage}}\\

\section{Research design}

\noindent Morbi rutrum odio eget arcu adipiscing sodales. Aenean et purus a est pulvinar pellentesque.Cras in elit neque,  quis varius elit.   Phasellus fringilla,  nibh eu tempus venenatis,  dolor elitposuere quam, quis adipiscing urna leo nec orci.  
\par Lorem ipsum dolor sit amet, consectetur adipiscing elit. Aliquam ultricies lacinia euismod. Nam tempus risus in dolor rhoncus in interdum enim tincidunt. Donec vel nunc neque. In condimentum ullamcorper quam non consequat. Fusce sagittis tempor feugiat. Fusce magna erat, molestie eu convallis ut, tempus sed arcu. Quisque molestie, ante a tincidunt ullamcorper, sapien enim dignissim lacus, in semper nibh erat lobortis purus. 

\par\vspace{0.5cm}
\noindent \Cshadowbox{
	\begin{minipage}{15cm}
		\bigskip
		\color[rgb]{0.0,0.4,0.65}This section provides a brief, high-level description of the broad research method to be used in carrying out the research. This research method will vary from field to field. The section should 
			\begin{itemize}
				\item identify the broad research method, and,
				\item provide a brief description of the method.
		\end{itemize} 
		\medskip
\end{minipage}}\\

%\section{Data}

%\noindent Sed ullamcorper quam eu nisl interdum at interdum enim egestas. Aliquam placerat justo sed lectus lobortis ut porta nisl porttitor. Vestibulum mi dolor, lacinia molestie gravida at, tempus vitae ligula. Donec eget quam sapien, in viverra eros.

%\par\vspace{0.5cm}
%\noindent \Cshadowbox{
%	\begin{minipage}{15cm}
%		\bigskip
%		\color[rgb]{0.0,0.4,0.65}This section deals with the data set(s) to be used in the research. It provides a description of these and discusses the pre-processing of the data set(s) that will be carried out for the research. It should include the following.
%			\begin{description}
%				\item[Source] Identify and describe the source of the data set(s).
%				\item[Data] Identify and describe the data set(s) themselves (features, records, etc.).
%				\item[Pre-processing] Identify and describe the pre-processing steps to be taken. 
%		\end{description} 
%		If your research does not involve data, this section can be omitted.
		\medskip
%end{minipage}}\\

\section{Methods}

\noindent Lorem ipsum dolor sit amet, consectetur adipiscing elit. Aliquam ultricies lacinia euismod. Nam tempus risus in dolor rhoncus in interdum enim tincidunt. Donec vel nunc neque. In condimentum ullamcorper quam non consequat. Fusce sagittis tempor feugiat. 

\par\vspace{0.4cm}
\noindent \Cshadowbox{
	\begin{minipage}{15cm}
		\medskip
		\color[rgb]{0.0,0.4,0.65}This section describes and \emph{motivates} the instruments and procedure to be used in carrying out the research. These should not be discussed in a chronological way (e.g. first, this step will be taken, and then this step will be taken, etc.), but should instead be grouped together systematically. A more detailed discussion of the relevant algorithms or models should be included here. The discussion should show an understanding of the \emph(available) methods and put forward a \emph{motivation for those chosen}. 
		
		%The section should include the following.
		%	\begin{description}
		%		\item[Instruments] Identify, e.g., software, architecture, algorithms, models, etc.
		%		\item[Decisions] Identify, e.g., training/test split, class imbalance, feature selection, etc.
		%		\item[Procedure] Identify, e.g., number of runs, prevention of data leakage, etc.
		%		\item[Data] Describe the data that will be generated by the methods.
		%\end{description} 
		\medskip
\end{minipage}}\\

%\section{Analysis}

%Morbi rutrum odio eget arcu adipiscing sodales. Aenean et purus a est pulvinar pellentesque.Cras in elit neque,  quis varius elit.   Phasellus fringilla,  nibh eu tempus venenatis,  dolor elitposuere quam, quis adipiscing urna leo nec orci.

%\par\vspace{0.4cm}
%\noindent \Cshadowbox{
%	\begin{minipage}{15cm}
%		\medskip
%		\color[rgb]{0.0,0.4,0.65}This section describes the analysis of the data that will be generated in carrying out the research. It should describe how you will determine the significance of your results. The section should include the following.
%			\begin{description}
%				\item[Descriptive Statistics] Identify, e.g., Mean, Standard Deviations, etc.
%				\item[Metrics] Identify, e.g., Mean Average Precision, Recall, etc.
%				\item[Baselines] Identify the relevant baselines in the literature.
%				\item[Comparisons] Identify, e.g., confusion matrices, AUC-ROC curve, etc. 
%		\end{description}
%		\medskip
%\end{minipage}}\\

\section{Limitations}

\noindent Sed ullamcorper quam eu nisl interdum at interdum enim egestas. Aliquam placerat justo sed lectus lobortis ut porta nisl porttitor. Vestibulum mi dolor, lacinia molestie gravida at, tempus vitae ligula. Donec eget quam sapien, in viverra eros. 

\par\vspace{0.4cm}
\noindent \Cshadowbox{
	\begin{minipage}{15cm}
		\medskip
		\color[rgb]{0.0,0.4,0.65}This section identifies limitations with the methodology. These limitations might have a number of sources. For example, constraints on time and/or computational power could limit the number of methods or models tested. In this section, you should identify and discuss any such limitations that are foreseeable at the time of proposal.
		\medskip
\end{minipage}}\\

\section{Ethical Considerations}

\noindent Lorem ipsum dolor sit amet, consectetur adipiscing elit. Aliquam ultricies lacinia euismod. Nam tempus risus in dolor rhoncus in interdum enim tincidunt. Donec vel nunc neque. In condimentum ullamcorper quam non consequat. Fusce sagittis tempor feugiat. Fusce magna erat, molestie eu convallis ut, tempus sed arcu. Quisque molestie, ante a tincidunt ullamcorper, sapien enim dignissim lacus, in semper nibh erat lobortis purus. Integer dapibus ligula ac risus convallis pellentesque.

\par\vspace{0.4cm}
\noindent \Cshadowbox{
	\begin{minipage}{15cm}
		\medskip
		\color[rgb]{0.0,0.4,0.65}This section assesses the need for ethical clearance for the proposed research.In cases in which no ethical clearance is required, the section will indicate as such and include a brief motivation for this. In cases in which ethical clearance is required, the section will indicate as such, include a brief motivation for this, and describe the procedure that will be followed for obtaining clearance.
		\medskip
\end{minipage}}\\


\chapter{Schedule of Work}
\label{Chapter3} % For referencing this chapter elsewhere, use \ref{Chapter4}


\noindent Lorem ipsum dolor sit amet, consectetur adipiscing elit. Aliquam ultricies lacinia euismod. Nam tempus risus in dolor rhoncus in interdum enim tincidunt. Donec vel nunc neque. In condimentum ullamcorper quam non consequat. Fusce sagittis tempor feugiat. 

\par\vspace{0.5cm}
\noindent \Cshadowbox{
	\begin{minipage}{15cm}
		\bigskip
		\color[rgb]{0.0,0.4,0.65}The chapter puts forward the planned schedule of work for the research. In it, you should
		this chapter, you should provide an overview and \emph{discussion} of the planned schedule for the research. Your discussion should 
		\begin{itemize}
			\item present and discuss the schedule of work,
			\item identify any foreseeable difficulties to carrying out the project according to the planned schedule,
			\item provide a brief conclusion to the chapter.
		\end{itemize} 
		\medskip
\end{minipage}}\\


\section{Schedule of Work}

Sed ullamcorper quam eu nisl interdum at interdum enim egestas. Aliquam placerat justo sed lectus lobortis ut porta nisl porttitor. Vestibulum mi dolor, lacinia molestie gravida at, tempus vitae ligula. Donec eget quam sapien, in viverra eros. Vivamus ornare ultrices facilisis. Ut hendrerit volutpat vulputate. Morbi condimentum venenatis augue, id porta ipsum vulputate in. Curabitur luctus tempus justo. Vestibulum risus lectus, adipiscing nec condimentum quis, condimentum nec nisl. Aliquam dictum sagittis velit sed iaculis. Morbi tristique augue sit amet nulla pulvinar id facilisis ligula mollis. Nam elit libero, tincidunt ut aliquam at, molestie in quam. Aenean rhoncus vehicula hendrerit. \par 
\medskip


\begin{ganttchart}[
	time slot format=isodate-yearmonth,
	time slot unit=month,
	]{2019-01}{2020-09}
	\gantttitlecalendar{year, month} \\
	\ganttgroup{Objective 1}{2019-01}{2019-11} \\
	\ganttbar{Task 1}{2019-02}{2019-08} \\
	\ganttbar{Task 2}{2019-02}{2019-11} \\
	\ganttmilestone{Milestone 1}{2019-11} \\
	\ganttgroup{Objective 2}{2019-12}{2020-08} \\
	\ganttbar{Task 3}{2019-12}{2020-08} \\
	\ganttbar{Final Task}{2020-06}{2020-08}
	\ganttlink[link type=s-s]{elem1}{elem2}
	\ganttlink[]{elem2}{elem3}
	\ganttlink[]{elem3}{elem5}	
	\ganttlink[link type=f-f]{elem5}{elem6}
	
\end{ganttchart}

\medskip

Lorem ipsum dolor sit amet, consectetur adipiscing elit. Aliquam ultricies lacinia euismod. Nam tempus risus in dolor rhoncus in interdum enim tincidunt. Donec vel nunc neque. In condimentum ullamcorper quam non consequat. Fusce sagittis tempor feugiat. Fusce magna erat, molestie eu convallis ut, tempus sed arcu. Quisque molestie, ante a tincidunt ullamcorper, sapien enim dignissim lacus, in semper nibh erat lobortis purus. Integer dapibus ligula ac risus convallis pellentesque. 

\section{Potential Difficulties}

Sed ullamcorper quam eu nisl interdum at interdum enim egestas. Aliquam placerat justo sed lectus lobortis ut porta nisl porttitor. Vestibulum mi dolor, lacinia molestie gravida at, tempus vitae ligula. Donec eget quam sapien, in viverra eros. Donec pellentesque justo a massa fringilla non vestibulum metus vestibulum. Vestibulum in orci quis felis tempor lacinia. Vivamus ornare ultrices facilisis. 



\chapter{Conclusion}
\label{Chapter4} % For referencing this chapter elsewhere, use \ref{Chapter4}

\noindent Lorem ipsum dolor sit amet, consectetur adipiscing elit. Aliquam ultricies lacinia euismod. Nam tempus risus in dolor rhoncus in interdum enim tincidunt. Donec vel nunc neque. In condimentum ullamcorper quam non consequat. Fusce sagittis tempor feugiat. Fusce magna erat, molestie eu convallis ut, tempus sed arcu. Quisque molestie, ante a tincidunt ullamcorper, sapien enim dignissim lacus, in semper nibh erat lobortis purus. Integer dapibus ligula ac risus convallis pellentesque.

\par\vspace{0.5cm}
\noindent \Cshadowbox{
	\begin{minipage}{15cm}
		\bigskip
		\color[rgb]{0.0,0.4,0.65}The final chapter provides a retrospective overview of the proposed research and is usually quite brief in relation to the rest of the report. A good way to structure this section is to proceed from the more specific to the more general, by
		\begin{itemize}
			\item reiterating the relevant gaps in the current literature,
			\item reiterating the aim of the research,
			\item emphasising the contribution that the proposed research will make to current research in the field.
		\end{itemize}
			\medskip
\end{minipage}}\\

%----------------------------------------------------------------------------------------
%	THESIS CONTENT - APPENDICES
%----------------------------------------------------------------------------------------

\appendix % Cue to tell LaTeX that the following "chapters" are Appendices

% Include the appendices of the thesis as separate files from the Appendices folder
% Uncomment the lines as you write the Appendices

\chapter{Appendix Title} % Main appendix title
\label{AppendixA} % For referencing this appendix elsewhere, use \ref{AppendixA}

\section{Main Section}

\noindent Lorem ipsum dolor sit amet, consectetur adipiscing elit. Aliquam ultricies lacinia euismod. Nam tempus risus in dolor rhoncus in interdum enim tincidunt. Donec vel nunc neque. In condimentum ullamcorper quam non consequat. Fusce sagittis tempor feugiat. Fusce magna erat, molestie eu convallis ut, tempus sed arcu.

\par\vspace{0.5cm}
\noindent \Cshadowbox{
	\begin{minipage}{15cm}
		\bigskip
		\color[rgb]{0.0,0.4,0.65} The appendices are sections in which complicated mathematical or other formulae, descriptions of experiments or apparatus, and any other specialised or lengthy material such as computer programme listings, copies of spectra or other instrumental outputs are found.  
		\medskip
\end{minipage}}\\
\chapter{Using this Template} % Main appendix title
\label{AppendixB} % For referencing this appendix elsewhere, use \ref{AppendixB}

\par\vspace{1cm}
\Cshadowbox{
	\begin{minipage}{15cm}
		\bigskip
		\color[rgb]{0.0,0.4,0.65}In the following appendix, some of the guidance from the original template is included. This includes the
			\begin{itemize}
				\item files and folders included in the template,
				\item guidance on filling the main.tex file with your information,  
				\item guidance on including references, tables, figures, and mathematical formulae.
			\end{itemize} 
		\bigskip
\end{minipage}}\\

\section{What this Template Includes}

\subsection{Folders}

This template comes as a single zip file that expands out to several files and folders. The folder names are mostly self-explanatory:

\keyword{Appendices} -- this is the folder where you put the appendices. Each appendix should go into its own separate \file{.tex} file. An example and template are included in the directory.

\keyword{Chapters} -- this is the folder where you put the thesis chapters. Each chapter should go in its own separate \file{.tex} file.

\keyword{Figures} -- this folder contains all figures for the thesis. These are the final images that will go into the thesis document.

\subsection{Files}

Included are also several files, most of them are plain text and you can see their contents in a text editor. After initial compilation, you will see that more auxiliary files are created by \LaTeX{} or BibTeX and which you don't need to delete or worry about:

\keyword{example.bib} -- this is an important file that contains all the bibliographic information and references that you will be citing in the thesis for use with BibTeX. You can write it manually, but there are reference manager programs available that will create and manage it for you. Bibliographies in \LaTeX{} are a large subject and you may need to read about BibTeX before starting with this. Many modern reference managers will allow you to export your references in BibTeX format which greatly eases the amount of work you have to do.

\keyword{MastersDoctoralThesis.cls} -- this is an important file. It is the class file that tells \LaTeX{} how to format the thesis. 

\keyword{main.pdf} -- this is your beautifully typeset thesis (in the PDF file format) created by \LaTeX{}. It is supplied in the PDF with the template and after you compile the template you should get an identical version.

\keyword{main.tex} -- this is an important file. This is the file that you tell \LaTeX{} to compile to produce your thesis as a PDF file. It contains the framework and constructs that tell \LaTeX{} how to layout the thesis. It is heavily commented so you can read exactly what each line of code does and why it is there. After you put your own information into the \emph{THESIS INFORMATION} block -- you have now started your thesis!

Files that are \emph{not} included, but are created by \LaTeX{} as auxiliary files include:

\keyword{main.aux} -- this is an auxiliary file generated by \LaTeX{}, if it is deleted \LaTeX{} simply regenerates it when you run the main \file{.tex} file.

\keyword{main.bbl} -- this is an auxiliary file generated by BibTeX, if it is deleted, BibTeX simply regenerates it when you run the \file{main.aux} file. Whereas the \file{.bib} file contains all the references you have, this \file{.bbl} file contains the references you have actually cited in the thesis and is used to build the bibliography section of the thesis.

\keyword{main.blg} -- this is an auxiliary file generated by BibTeX, if it is deleted BibTeX simply regenerates it when you run the main \file{.aux} file.

\keyword{main.lof} -- this is an auxiliary file generated by \LaTeX{}, if it is deleted \LaTeX{} simply regenerates it when you run the main \file{.tex} file. It tells \LaTeX{} how to build the \emph{List of Figures} section.

\keyword{main.log} -- this is an auxiliary file generated by \LaTeX{}, if it is deleted \LaTeX{} simply regenerates it when you run the main \file{.tex} file. It contains messages from \LaTeX{}, if you receive errors and warnings from \LaTeX{}, they will be in this \file{.log} file.

\keyword{main.lot} -- this is an auxiliary file generated by \LaTeX{}, if it is deleted \LaTeX{} simply regenerates it when you run the main \file{.tex} file. It tells \LaTeX{} how to build the \emph{List of Tables} section.

\keyword{main.out} -- this is an auxiliary file generated by \LaTeX{}, if it is deleted \LaTeX{} simply regenerates it when you run the main \file{.tex} file.

So from this long list, only the files with the \file{.bib}, \file{.cls} and \file{.tex} extensions are the most important ones. The other auxiliary files can be ignored or deleted as \LaTeX{} and BibTeX will regenerate them.

%----------------------------------------------------------------------------------------

\section{Filling in Your Information in the \file{main.tex} File}\label{FillingFile}

You will need to personalise the thesis template and make it your own by filling in your own information. This is done by editing the \file{main.tex} file in a text editor or your favourite LaTeX environment.

Open the file and scroll down to the third large block titled \emph{THESIS INFORMATION} where you can see the entries for \emph{University Name}, \emph{Department Name}, etc \ldots

Fill out the information about yourself and institution. You can also insert web links, if you do, make sure you use the full URL, including the \code{http://} for this. If you don't want these to be linked, simply remove the \verb|\href{url}{name}| and only leave the name.

When you have done this, save the file and recompile \code{main.tex}. All the information you filled in should now be in the PDF, complete with web links. You can now begin your thesis proper!

%----------------------------------------------------------------------------------------

\section{Thesis Features and Conventions}\label{ThesisConventions}

To get the best out of this template, there are a few conventions that you may want to follow.

One of the most important (and most difficult) things to keep track of in such a long document as a thesis is consistency. Using certain conventions and ways of doing things (such as using a Todo list) makes the job easier. Of course, all of these are optional and you can adopt your own method.

\subsection{References}

The \code{biblatex} package is used to format the bibliography and inserts references such as this one \parencite{Reference1}. The options used in the \file{main.tex} file mean that the in-text citations of references are formatted with the author(s) listed with the date of the publication. Multiple references are separated by semicolons (e.g. \parencite{Reference2, Reference1}) and references with more than three authors only show the first author with \emph{et al.} indicating there are more authors (e.g. \parencite{Reference3}). This is done automatically for you. To see how you use references, have a look at the \file{Chapter1.tex} source file. Many reference managers allow you to simply drag the reference into the document as you type.

Scientific references should come \emph{before} the punctuation mark if there is one (such as a comma or period). The same goes for footnotes\footnote{Such as this footnote, here down at the bottom of the page.}. You can change this but the most important thing is to keep the convention consistent throughout the thesis. Footnotes themselves should be full, descriptive sentences (beginning with a capital letter and ending with a full stop). The APA6 states: \enquote{Footnote numbers should be superscripted, [...], following any punctuation mark except a dash.} The Chicago manual of style states: \enquote{A note number should be placed at the end of a sentence or clause. The number follows any punctuation mark except the dash, which it precedes. It follows a closing parenthesis.}

The bibliography is typeset with references listed in alphabetical order by the first author's last name. This is similar to the APA referencing style. To see how \LaTeX{} typesets the bibliography, have a look at the very end of this document (or just click on the reference number links in in-text citations).

\subsubsection{A Note on bibtex}

The bibtex backend used in the template by default does not correctly handle unicode character encoding (i.e. "international" characters). You may see a warning about this in the compilation log and, if your references contain unicode characters, they may not show up correctly or at all. The solution to this is to use the biber backend instead of the outdated bibtex backend. This is done by finding this in \file{main.tex}: \option{backend=bibtex} and changing it to \option{backend=biber}. You will then need to delete all auxiliary BibTeX files and navigate to the template directory in your terminal (command prompt). Once there, simply type \code{biber main} and biber will compile your bibliography. You can then compile \file{main.tex} as normal and your bibliography will be updated. An alternative is to set up your LaTeX editor to compile with biber instead of bibtex, see \href{http://tex.stackexchange.com/questions/154751/biblatex-with-biber-configuring-my-editor-to-avoid-undefined-citations/}{here} for how to do this for various editors.

\subsection{Tables}

Tables are an important way of displaying your results, below is an example table which was generated with this code:

{\small
\begin{verbatim}
\begin{table}
\caption{The effects of treatments X and Y on the four groups studied.}
\label{tab:treatments}
\centering
\begin{tabular}{l l l}
\toprule
\tabhead{Groups} & \tabhead{Treatment X} & \tabhead{Treatment Y} \\
\midrule
1 & 0.2 & 0.8\\
2 & 0.17 & 0.7\\
3 & 0.24 & 0.75\\
4 & 0.68 & 0.3\\
\bottomrule\\
\end{tabular}
\end{table}
\end{verbatim}
}

\begin{table}
\caption{The effects of treatments X and Y on the four groups studied.}
\label{tab:treatments}
\centering
\begin{tabular}{l l l}
\toprule
\tabhead{Groups} & \tabhead{Treatment X} & \tabhead{Treatment Y} \\
\midrule
1 & 0.2 & 0.8\\
2 & 0.17 & 0.7\\
3 & 0.24 & 0.75\\
4 & 0.68 & 0.3\\
\bottomrule\\
\end{tabular}
\end{table}

You can reference tables with \verb|\ref{<label>}| where the label is defined within the table environment. See \file{Chapter1.tex} for an example of the label and citation (e.g. Table~\ref{tab:treatments}).

\subsection{Figures}

There will hopefully be many figures in your thesis (that should be placed in the \emph{Figures} folder). The way to insert figures into your thesis is to use a code template like this:
\begin{verbatim}
\begin{figure}
\centering
\includegraphics{Figures/Electron}
\decoRule
\caption[An Electron]{An electron (artist's impression).}
\label{fig:Electron}
\end{figure}
\end{verbatim}
Also look in the source file. Putting this code into the source file produces the picture of the electron that you can see in the figure below.

\begin{figure}[th]
\centering
\includegraphics{Figures/Electron}
\decoRule
\caption[An Electron]{An electron (artist's impression).}
\label{fig:Electron}
\end{figure}

Sometimes figures don't always appear where you write them in the source. The placement depends on how much space there is on the page for the figure. Sometimes there is not enough room to fit a figure directly where it should go (in relation to the text) and so \LaTeX{} puts it at the top of the next page. Positioning figures is the job of \LaTeX{} and so you should only worry about making them look good!

Figures usually should have captions just in case you need to refer to them (such as in Figure~\ref{fig:Electron}). The \verb|\caption| command contains two parts, the first part, inside the square brackets is the title that will appear in the \emph{List of Figures}, and so should be short. The second part in the curly brackets should contain the longer and more descriptive caption text.

The \verb|\decoRule| command is optional and simply puts an aesthetic horizontal line below the image. If you do this for one image, do it for all of them.

\LaTeX{} is capable of using images in pdf, jpg and png format.

\subsection{Typesetting mathematics}

If your thesis is going to contain heavy mathematical content, be sure that \LaTeX{} will make it look beautiful, even though it won't be able to solve the equations for you.

The \enquote{Not So Short Introduction to \LaTeX} (available on \href{http://www.ctan.org/tex-archive/info/lshort/english/lshort.pdf}{CTAN}) should tell you everything you need to know for most cases of typesetting mathematics. If you need more information, a much more thorough mathematical guide is available from the AMS called, \enquote{A Short Math Guide to \LaTeX} and can be downloaded from:
\url{ftp://ftp.ams.org/pub/tex/doc/amsmath/short-math-guide.pdf}

There are many different \LaTeX{} symbols to remember, luckily you can find the most common symbols in \href{http://ctan.org/pkg/comprehensive}{The Comprehensive \LaTeX~Symbol List}.

You can write an equation, which is automatically given an equation number by \LaTeX{} like this:
\begin{verbatim}
\begin{equation}
E = mc^{2}
\label{eqn:Einstein}
\end{equation}
\end{verbatim}

This will produce Einstein's famous energy-matter equivalence equation:
\begin{equation}
E = mc^{2}
\label{eqn:Einstein}
\end{equation}

All equations you write (which are not in the middle of paragraph text) are automatically given equation numbers by \LaTeX{}. If you don't want a particular equation numbered, use the unnumbered form:
\begin{verbatim}
\[ a^{2}=4 \]
\end{verbatim}

\begin{flushright}
Guide written by ---\\
Sunil Patel: \href{http://www.sunilpatel.co.uk}{www.sunilpatel.co.uk}\\
Vel: \href{http://www.LaTeXTemplates.com}{LaTeXTemplates.com}
\end{flushright}


%\include{Appendices/AppendixC}

%----------------------------------------------------------------------------------------
%	BIBLIOGRAPHY
%----------------------------------------------------------------------------------------

\printbibliography[heading=bibintoc]

%----------------------------------------------------------------------------------------

\end{document}  
